\documentclass[a4paper,12pt]{article}

\input{preambula}

\begin{document}
\begin{center}
	\textbf{Игра по типу сороконжки}
\end{center}
Двое играют в следующую простую разновидность покера.

Сначала первый игрок берёт карту из колоды (карта с равными вероятностями может быть красной или чёрной), смотрит на неё. После чего может вскрыться или удвоить ставку. Если ставка удвоена, то второй либо просит показать карту, либо сбрасывает её обратно в колоду. Если карта чёрная, то первый платит второму (1 до удвоения или 2 после удвоения), если красная, то, наоборот, второй платит первому 1 или 2 в зависимости наличия удвоения. Если карта сброшена, то второй игрок платит первому рубль.

\begin{enumerate}
	\item \textbf{Как выглядит дерево этой игры?}\\
	\textit{Ответ:\\
	\begin{figure}[h]
		\centering
		\includegraphics[width=0.7\linewidth]{wrq-xCfeEeaNKg4a0RSP6Q_fcfea7413b8afe48a6a40361b7de01b6_w06_test2}
		\caption{}
		\label{fig:wrq-xcfeeeankg4a0rsp6qfcfea7413b8afe48a6a40361b7de01b6w06test2}
	\end{figure}
	Вначале природа выбирает, какая карта выпадет. Первый игрок знает о том, какая карта ему досталась, и решает, удвоить ставку или нет. В свою очередь, второй игрок не знает, какая карта у первого, и решает, вскрываться или нет. Поэтому у второго есть информационное множество.}
	\item Предположим, что оба игрока используют смешанные стратегии в управляемых им вершинах, кроме того, второй игрок верит в то, что находится в верхней вершине с вероятностью $\omega$. Вероятности и веры согласованы между собой. Допустим, что первый игрок с ненулевой вероятностью удваивает в обоих случаях, и данные профили стратегий находятся в равновесии.
	
	\textbf{Какое соотношение на веру $\omega$ второго игрока о нахождении его в верхней вершине выполнено?}\\
	\textit{Ответ: $$2\omega-2(1-\omega) = -1$$ Если второй игрок верит, что он находится в верхней вершине с вероятностью $\omega$, и в нижней с вероятностью $1-\omega$ и использует смешанное равновесие, то ему всё равно, как сходить в этом случае. Ожидаемый выигрыш от применения стратегии "вскрыть карты" равен $2\omega-2(1-\omega)$, от стратегии сбросить - $-1$.}
	\item \textbf{Чему равна вера $\omega$ второго игрока о нахождении его в верхней вершине?}\\
	\textit{Ответ: из предыдущего $\omega = 0.25$.}
	\item Будем считать, что мы находимся в предположениях задачи 2. Пусть вероятность вскрыть карты у второго игрока равна $\gamma$, а вероятности повысить удвоить у первого игрока равны $\alpha$ при чёрной карте и $\beta$ при красной карте (см. рисунок внизу после заданий).
	\begin{figure}[h!]
		\centering
		\includegraphics[width=0.7\linewidth]{wVBHNifgEeak5wpS_JmWWw_6fbbaac5a5e6813956346cf1ffe95c35_w06_test5}
		\caption{}
		\label{fig:wvbhnifgeeak5wpsjmwww6fbbaac5a5e6813956346cf1ffe95c35w06test5}
	\end{figure}

	\textbf{При каком $\gamma$ первый игрок будет смешивать свои стратегии при чёрной и красной картах?}\\
	\textit{Ответ: $$(2/3, 0)$$ Первый игрок будет смешивать свои стратегии, если ожидаемый выигрыш от использования стратегий "удвоить" или "вскрыть" будет одинаковым. Поэтому мы получаем равенства: $-1=-2\gamma+(1-\gamma)$ для чёрной карты и $1=2\gamma+(1-\gamma)$ для красной карты, откуда $\gamma=2/3$ в первом случае и $\gamma=0$ во втором.}
	\item Исходя из предыдущей задачи, мы получаем, что первый игрок не будет смешивать свои стратегии в обоих случаях. Так как мы рассматриваем случай, когда он удваивает в каждом из случаев с ненулевой вероятностью логично предположить, что $\gamma=2/3$. \textbf{Какие значения принимают $\alpha$ и $\beta$ (Напоминаем, что вера $\omega$ второго игрока должна быть согласована с вероятностями попасть в соответствующую вершину.)}\\
	\textit{Ответ: $\alpha = 1/3$ (объяснение ниже). }
	\item \textit{$\beta = 1$. Если $\gamma=2/3$, то $\beta = 1$, так как первому игроку в случае удачной карты будет выгоднее удвоить ставку. Так как вера $\omega = 1/4$, то мы имеем соотношение: $\frac{1/2\alpha}{1/2\beta} = \frac{\omega}{1-\omega}$, откуда $\alpha = 1/3$.}
\end{enumerate}


\end{document}