\documentclass[a4paper,12pt]{article}
\include{preambula}

\renewcommand{\baselinestretch}{1.3}

\title{vopr}
\author{Георгий Демьянов}
\date{today}
\usepackage[left=1.27cm,right=1.27cm,top=1.27cm,bottom=2cm]{geometry}

\begin{document}
\begin{center}
	\textbf{Вопрос по выбору}
\end{center}	

Пусть есть поршень, который находится в термостате с температурой $T_0$ и с бесконечной теплоемкостью.

Поставим вопрос: \textbf{возможно ли построить тепловой двигатель, имея один источник тепла?}
\begin{figure}[h!]
	\centering
	\includegraphics[width=0.3\linewidth]{ris1}
	\caption{Поршень в термостате}
	\label{fig:ust}
\end{figure}

Тогда на $PV$--диаграмме можно изобразить только изотерму.
\begin{figure}[h!]
	\centering
	\includegraphics[width=0.25\linewidth]{ris2}
	\caption{$PV$--диаграмма}
	\label{fig:ust}
\end{figure}
Если мы будем двигаться вдоль по изотерме <<вперед--назад>>, то полезная работа будет равна нулю, т.е. смысла в такой машине нет.

Что же можно сделать еще? Если наш поршень находится в адиабатической оболочке, то мы можем <<бегать>> по адиабате, для чего нужно совершать достаточно быстрые движения поршнем.
\begin{figure}[h!]
	\centering
	\includegraphics[width=0.25\linewidth]{ris3}
	\caption{$PV$--диаграмма}
	\label{fig:ust}
\end{figure}
В нашем же случае попытаемся построить такой процесс, который был бы чем-то средним между изотермой и адиабатой. Тогда нужно двигать поршень не бесконечно медленно (т.е. <<бегать>> не по изотерме), а немного быстрее, но не так быстро, чтобы попасть на адиабату. Тогда будет существовать теплообмен между идеальным газом и термостатом. При этом, будем утверждать, что газ в каждый момент времени будет достаточно отрелаксированным.

Для того, чтобы построить такой процесс, найдем зависимость $\Delta P = \Delta P(Q, \Delta V)$.

Запишем первое начало термодинамики в приращениях:
\begin{equation}
C_V \Delta T = Q - P\Delta V,
\end{equation}
где $C_V$ --- теплоемкость газа при постоянном объеме, $\Delta T$ --- малые изменения температуры, $\Delta V$ --- малые изменения объема, $Q$ --- тепло, $P$ --- давление газа.

Запишем уравнение состояния идеального газа для 1 моля в приращениях:
\begin{equation}
P\Delta V+V\Delta P = R\Delta T.
\end{equation}

Решая систему из этих уравнений, получим:
\begin{equation}
\boxed{
\Delta P = \cfrac{R}{VC_V}\,Q-\cfrac{\gamma P}{V}\,\Delta V},
\label{main}
\end{equation}
где $\gamma$ --- показатель адиабаты.

Т.о. мы получили зависимость изменения давления идеального газа, который совершает работу $P\Delta V$ и обменивается теплом $Q$ с термостатом. Заметим, что при $Q=0$ получаем $\Delta P_{\text{ад}} = -\frac{\gamma P}{V}\,\Delta V$ --- изменение давления в адиабатическом процессе. Тогда запишем уравнение \eqref{main} в кратком виде:
\begin{equation}
\boxed{
\Delta P = \cfrac{R}{VC_V}\,Q+\Delta P_{\text{ад}}}.
\label{main2}
\end{equation}

Построим $PV$--диаграммы (рис. \ref{fig:res}).\\
\begin{figure}[h!]
	\centering
	\includegraphics[width=0.9\linewidth]{ris4}
	\caption{$PV$--диаграмма}
	\label{fig:res}
\end{figure}
Рассмотрим два случая:\\
I. $Q>0$, т.е. $T_{\text{г}} < T_0$ (слева);\\
II. $Q<0$, т.е. $T_{\text{г}} > T_0$ (справа).\\
На диаграммах пунктиром изображено семейство адиабат, изотерма с температурой $T_0$. 

\begin{enumerate}
	\item[I.] \begin{enumerate}
		\item[1)] Рассмотрим расширение газа по адиабате ($1\rightarrow 2$). $Q>0$, $\Delta P_{\text{ад}}<0$, отсюда в соответствии с формулой \eqref{main2} $\abs{\Delta P}<\abs{\Delta P_{\text{ад}}}$. Таким образом, мы будем наблюдать <<загибание процесса>> ($1\rightarrow 2'$ или $1\rightarrow 2''$).
		\item[2)] Рассмотрим сжатие газа по адиабате ($1\rightarrow 2$). $Q>0$, $\Delta P_{\text{ад}}>0$, отсюда в соответствии с формулой \eqref{main2} $\abs{\Delta P}>\abs{\Delta P_{\text{ад}}}$. Таким образом, наклон графика процесса выглядит <<круче>>, чем в адиабатическом процессе ($1\rightarrow 2'$ или $1\rightarrow 2''$).
	\end{enumerate}
	Можно наблюдать, что добавка теплоты $Q$ приближает процесс к изотерме $T_0$.
	\item[II.] \begin{enumerate}
		\item[1)] Рассмотрим расширение газа по адиабате ($1\rightarrow 2$). $Q<0$, $\Delta P_{\text{ад}}<0$, отсюда в соответствии с формулой \eqref{main2} $\abs{\Delta P}>\abs{\Delta P_{\text{ад}}}$. Таким образом, процесс <<будет идти>> стремительнее ($1\rightarrow 2'$).
		\item[2)] Рассмотрим сжатие газа по адиабате ($1\rightarrow 2$). $Q<0$, $\Delta P_{\text{ад}}>0$, отсюда в соответствии с формулой \eqref{main2} $\abs{\Delta P}<\abs{\Delta P_{\text{ад}}}$. Таким образом, процесс <<будет идти>> менее <<круто>> ($1\rightarrow 2'$).
	\end{enumerate}
\end{enumerate}
В итоге получаем, что все процессы стремятся к изотерме $T_0$.

А теперь попробуем построить круговой процесс.
\begin{figure}[h!]
	\centering
	\includegraphics[width=0.3\linewidth]{ris5}
	\caption{Круговой процесс}
	\label{fig:ust}
\end{figure}

Таким образом, \textit{мы получили ход против часовой стрелки} ($A<0$), причем этот \textit{процесс необратим}. Обратного хода быть не может, т.к. есть теплообмен газа и резервуара.

Заметим, что чем медленнее будем двигаться, там <<уже>> будет картинка, т.е. мы приближаемся к изотерме. Чем быстрее будем двигаться, тем картинка <<шире>>, т.е. процесс приближается к адиабатическому.

Мы получили утилизатор работы.

\textbf{Вывод:} \textit{имея один тепловой резервуар, нельзя построить тепловую машину, но можно построить утилизатор работы.}

Отсюда легко получить \textbf{неравенство Клаузиуса}:

Пусть мы имеем один тепловой резервуар. Газ можно перевести из состояния 1 в состояние 2 двумя способами: обратимым и необратимым.
\begin{figure}[h!]
	\centering
	\includegraphics[width=0.2\linewidth]{ris6}
	%\caption{Круговой процесс}
	\label{fig:ust}
\end{figure}

\begin{center}
$
\delta Q^{\text{ноб}}=dU+\delta A^{\text{ноб}}$\\
$\delta Q^{\text{об}}=dU+\delta A^{\text{об}}
$
\end{center}
Из этого можно получить круговой процесс путем обращения обратимого процесса. Тогда:
\begin{center}
$\delta Q = \delta Q^{\text{ноб}}-\delta Q^{\text{об}} = \delta A^{\text{ноб}}-\delta A^{\text{об}} \leq 0$ --- по доказанному\\
$\delta Q^{\text{ноб}}\leq\delta Q^{\text{об}} = TdS$\\
$dS\geq \cfrac{\delta Q^{\text{ноб}}}{T}$
\vspace{-0.5cm}
$$0=\oint\limits dS\geq \oint{\cfrac{\delta Q^{\text{ноб}}}{T}}$$
$$\boxed{\oint{\cfrac{\delta Q^{\text{ноб}}}{T}}\leq0}\text{ --- \textbf{неравенство Клаузиуса.}}$$
\end{center}

Отсюда следует, что в необратимых адиабатически--изолированных процессах энтропия не убывает --- \textbf{закон возрастания энтропии}:
$$
dS\geq \cfrac{\delta Q^{\text{ноб}}}{T}=0 \Rightarrow dS\geq 0.
$$








\begin{center}
	\vfill \emph{{\small Г. С. Демьянов, 642 группа, 2017 г.}}
\end{center}
\end{document}