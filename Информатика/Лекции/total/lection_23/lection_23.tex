\chapter{Двоичное дерево поиска. Продолжение.}
\section{Класс Дерево. Продолжение.}
%Будет разумным погрузить класс звена в дерево
\begin{infa}{Класс Дерево}
\ \num \classi Tree:
\ \num \tab \classi Node: \com{Класс в классе можно делать, т.к. в классе описано свое\\\phantom{12\ \ \ \ \ \ \ } пространство имен}
\ \num \tab \tab \defi __init__(self, data):
\ \num \tab \tab \tab self.parent = \Nonei
\ \num \tab \tab \tab self.left = \Nonei
\ \num \tab \tab \tab self.right = \Nonei
\ \num \tab \tab \tab self.key = data
\ \num \tab \defi __init__(self):
\ \num \tab \tab self.root = \Nonei
\num \tab \defi find(self, data):
\num \tab \tab p = self.root
\num \tab \tab \whilei p \isi \noti \Nonei \andi p.key != data: \com{Важна последовательность\\\phantom{12\ \ \ \ \ \ \ } написаний условий, т.к. может быть ошибка при проверке ключа}
\num \tab \tab \tab \ifi data > p.key:
\num \tab \tab \tab \tab p = p.right
\num \tab \tab \tab \elsei:
\num \tab \tab \tab \tab p = p.left
\num \tab \tab \returni p
\num \tab \defi insert(self, data):
\num \tab \tab p = self.find(data) \com{Одно и то же число не может храниться дважды, \\\phantom{12\ \ \ \ \ \ \ }как в множестве, но значения могут повторяться}
\num \tab \tab \ifi p \isi \noti \Nonei:
\num \tab \tab \tab \returni
\num \tab \tab node = Tree.Node(data)
\num \tab \tab \ifi self.root \isi \Nonei:
\num \tab \tab \tab self.root = node
\num \tab \tab \tab \returni
\num \tab \tab p = self.root
\num \tab \tab \whilei \Truei:
\num \tab \tab \tab \ifi data < p.key:
\num \tab \tab \tab \tab \ifi p.left \isi \Nonei:
\num \tab \tab \tab \tab \tab p.left = node
\num \tab \tab \tab \tab \tab node.parent = p
\num \tab \tab \tab \tab \tab \breaki
\num \tab \tab \tab \tab \elsei:
\num \tab \tab \tab \tab \tab p = p.left 
\num \tab \tab \tab \elsei:
\num \tab \tab \tab \tab \ifi p.right \isi \Nonei:
\num \tab \tab \tab \tab \tab p.right = node
\num \tab \tab \tab \tab \tab node.parent = p
\num \tab \tab \tab \tab \tab \breaki
\num \tab \tab \tab \tab \elsei:
\num \tab \tab \tab \tab \tab p = p.right
\end{infa}
\section{Балансировка дерева}
Двоичное дерево поиска \textsf{сбалансированно}, если для каждой его вершины высота левого и правого поддерева отличаются не более чем на единицу.
	
%нарисуем закошеное дерево 
\textbf{Инвариант}: та вершина, которая левее других вершин, должна остаться левее всех вершин, т.е двигать вверх--вниз вершины можно, но влево--вправо двигать нельзя, иначе нарушится последовательность чисел.

Алгоритм балансировки подробно описан в \href{https://ru.wikipedia.org/wiki/%D0%90%D0%92%D0%9B-%D0%B4%D0%B5%D1%80%D0%B5%D0%B2%D0%BE#.D0.91.D0.B0.D0.BB.D0.B0.D0.BD.D1.81.D0.B8.D1.80.D0.BE.D0.B2.D0.BA.D0.B0}{википедии}.

\textsf{АВЛ--дерево} --- сбалансированное по высоте двоичное дерево поиска: для каждой его вершины высота её двух поддеревьев различается не более чем на 1.


\textsf{Красно--чёрное дерево} --- это одно из самобалансирующихся двоичных деревьев поиска, гарантирующих логарифмический рост высоты дерева от числа узлов и быстро выполняющее основные операции дерева поиска: добавление, удаление и поиск узла. Сбалансированность достигается за счёт введения дополнительного атрибута узла дерева --- «цвета». Этот атрибут может принимать одно из двух возможных значений --- «чёрный» или «красный».

\textbf{Свойства} красно--черного дерева:
\begin{enumerate}
	\item Узел либо красный, либо чёрный.
	\item Корень --- чёрный. (В других определениях это правило иногда опускается. Это правило слабо влияет на анализ, так как корень всегда может быть изменен с красного на чёрный, но не обязательно наоборот).
	\item Все листья --- чёрные.
	\item Оба потомка каждого красного узла --- чёрные.
	\item Всякий простой путь от данного узла до любого листового узла, являющегося его потомком, содержит одинаковое число чёрных узлов.
\end{enumerate}
