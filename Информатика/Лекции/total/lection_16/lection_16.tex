\chapter{Повторение}
\section{Ссылочная модель данных}
Ссылочная модель данных - объекты существуют независимо от времен.
\begin{alltt}
>>> 2+3
5
\end{alltt}
Создаются два объекта типа int.
\begin{alltt}
2.__add__(3) - метод добавления
\end{alltt}
\begin{alltt}
	x = 2+3 # x ссылается на объект 5
	x = 'Hello' # возникает объект строки
\end{alltt}
После вывода результата объекты 2, 3, 5 удаляются, т.к. нет ссылки на эти объекты.
Если x перестал ссылаться на 5, объект 5 уничтожается.
\begin{alltt}
x = 5
y = x
x = 'Hello'
y = None
\end{alltt}
После того, как y начал ссылаться на None, объект 5 уничтожается.
\begin{alltt}
	x = (2+3+5)**5**5 # приоритет у возведения в степени
\end{alltt}

Т.е. в начале выполняется 5**5, а потом (2+3+5)**25.

Начиная с версии Python 3.6, можно разделять числа так:
\begin{alltt}
x = 100_001
y = 0xAF_DC
\end{alltt}
\section{Пространство имен}
4 пространства: локальные, окружающие, глобальные, строенные (LEGB).

y = 10*x+7
Сначала будет происходить поиск локального x, затем в надпространстве, затем в глобальных, в последнюю очередь - во встроенных переменных.

Если написать 
max = 10,
то функция max перестанет работать.

Пример:
\begin{infa}{}
\num \defi f(A):
\num \tab A = A + 10 \com{если написать A += 10 ошибки не будет}
\num B = [1, 2, 3]
\num f(B)
\num \printi(*B) \com{1 2 3}
\end{infa}

Строки и числа --- неизменяемые объекты.
f является именем объекта functional.

def --- по сути это операция создания нового объекта.

Любой вызов функции порождает свое пространство имен, которое перестанет существовать после выполнения return. 

A начинает ссылаться на [1, 2, 3]. После конкатинации A начинает ссылаться на [1, 2, 3, 4]. A ссылается на глобальный объект, и начинает ссылаться на локальный объект. После return уничтожается A и список [1, 2, 3, 4].

Или можно записать так:
\begin{infa}{}
	\num \defi f(A):
	\num \tab A.append(10)
	\num B = [1, 2, 3]
	\num f(B)
	\num \printi(*B) \com{1 2 3}
\end{infa}

Функция должна что-то возвращать. В частном случае, можно возвращать несколько параметров. Нарушить ссылочную модель можно только "залезая"\ в глобальные имена.

\begin{infa}{}
	\num d=1
	\num \defi f(A):
	\num \tab \globali d
	\num \tab A.append(d)
	\num \tab d = d + 1
\end{infa}







\section{ООП}
Тип тоже является объектом типа тип. 
\begin{infa}{}
\num \classi Base:
\num \tab x = 10
\num \tab \defi g(self, x0):
\num \tab \tab self.x = x0
\num b = Base()
\num \printi(b.x)
\end{infa}

\begin{infa}{}
	\num \classi Base:
	\num \tab x = 10
	\num \tab \defi g(self, x0)
	\num \tab \tab self.x += x0
	\num b = Base()
	\num \printi(b.x)
\end{infa}

Нужно создавать атрибуты только в методе init!

\section{Элементы функционального программирования}

\subsection{Функция map}
\begin{alltt}
x, y, z = map(int, input().split()) \com{считывание трех чисел с клавиатуры}
\end{alltt}

Функция map применяет к каждому объекту функцию, которую мы написали. В нее же можно написать свою функцию:
\begin{alltt}
x, y, z = map(lambda x: int(x)**2, input().split())
	A == list(map(int, range(100)))
	B = map(float, int(x) for x in input().split())
\end{alltt}

map возвращает объект типа map.

\subsection{Применение lambda-функций}
$x^2+e^{1/x}+\ln x$
\begin{alltt}
x = decompoused_value # нужно объявлять функцию
(lambda x: x**2 + exp(1/x)+ln(x))(2) #хороший пример использования lambda
\end{alltt}

\subsection{Функция enumerate}
\begin{infa}{}
\num A = [10, 20, 30]
\num \fori i,x \ini \enumeratei(A): \com{А используется как итерируемый объект}
\num \tab \printi(i,x) \com{(0,10), (1,20), (2,30) - эту конструкцию нам вернет enumerate(A)}
\num \tab x = x + 1	\com{Значение в массиве не изменилось, мы только испортили х}
\end{infa}
\subsection{Функция zip}

\begin{alltt}
A = [1, 2, 3, 4, 5]
B = 'Hello'
c = list(zip(A,B)) \com{результат есть zip-object}
\end{alltt} 