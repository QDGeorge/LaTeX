\begin{enumerate}
	\item{
		$\lambda =1   $ \\ 
		$
		\begin{pmatrix*}[r]
		2 & 1 & -2 & \vrule & 0\\
		2 & 1 & -2 & \vrule & 0\\
		2 & 1 & -2 & \vrule & 0\\
		\end{pmatrix*} ,
		$
		$
		L_1=
		\left(
		\begin{smallmatrix*}[r]
		x_1\\ x_2\\ x_3\\ 
		\end{smallmatrix*}
		\right) 
		=
		\langle
		\underbrace{
			\left(
			\begin{smallmatrix*}[r] % в матрицах лучше не пользоваться frac
			-1/2\\ -1\\ 0\\ 
			\end{smallmatrix*}
			\right) ,
			\left(
			\begin{smallmatrix*}[r]
			1\\ 0\\ 1\\ 
			\end{smallmatrix*}
			\right) 
		}_{\substack{\text{формирует}\\\text{плоскость}}}
		\rangle
		$\\
		$ \dim L_1=2$ --- геометрическая кратность (размерность собственного подпространства). % мат функции
	}
	\item $\lambda = 2$ $$\Longrightarrow
	\begin{pmatrix}
	1 & 1 & -2 & \vrule & 0\\
	2 & 0 & -2 & \vrule & 0\\
	2 & 1 & -3 & \vrule & 0\\
	\end{pmatrix}
	\Rightarrow
	~L_1 = 
	\langle
	\left(
	\begin{smallmatrix*}[c]
	x_1\\ x_2\\ x_3\\ 
	\end{smallmatrix*}
	\right) 
	\rangle=
	\langle
	\left(
	\begin{smallmatrix*}[c]
	1\\ 1\\ 1\\ 
	\end{smallmatrix*}
	\right) 
	\rangle $$
\end{enumerate}
	Выберем базис: $\left\{
		\left(
	\begin{smallmatrix*}[c]
	-1/2\\ 1\\ 0\\
	\end{smallmatrix*}
		\right) 
	,
		\left(
	\begin{smallmatrix*}[c]
	1\\ 0\\ 1\\
	\end{smallmatrix*}
		\right) 
	,
		\left(
	\begin{smallmatrix*}[c]
	1\\ 1\\ 1\\
	\end{smallmatrix*}
		\right) 
	\right\}$\\
	
	Поэтому $\varphi(\textbf{f$_1$})=\textbf{f$_1$},~\varphi(\textbf{f$_2$})=\textbf{f$_2$},~\varphi(\textbf{\textbf{f$_3$}})=2\textbf{f$_3$}$\\ % ВЕКТОРЫ!
	
\textbf{Ответ:} $A'=\begin{pmatrix}
1 & 0 & 0\\
0 & 1 & 0\\
0 & 0 & 2\\
\end{pmatrix}.$

	\bigskip
	
	$\bullet$ Геометрическая кратность $\le$ алгебраическая кратность\\
	
	$\bullet$ Если геометрическая кратность строго меньше ($<$) алгебраической кратности хотя бы для одного $\lambda$, то преобразование \textsf{недиагонализируемо}.
	
	\begin{prim}
Диагонализировать матрицу:~~$A=\begin{pmatrix}
2 & 1 & 0\\
0 & 2 & 1\\
0 & 0 & 2\\
\end{pmatrix}$\\
	\end{prim}
	
$$
\det~(A - \lambda E) = 0 \Rightarrow
\begin{vmatrix*}[c]
        2-\lambda & 1 & 0\\
        0 & 2-\lambda  & 1\\
        0 & 0  & 2-\lambda\\
\end{vmatrix*}
= (2-\lambda)^3= 0 $$

Получаем, что $\lambda = 2$ алгебраической кратности 3.

$$\Longrightarrow
\begin{pmatrix}
0 & 1 & 0 & \vrule & 0\\
0 & 0 & 1 & \vrule & 0\\
0 & 0 & 0 & \vrule & 0\\
\end{pmatrix};
~L_1 = 
\langle
\left(
\begin{smallmatrix*}[c]
x_1\\ x_2\\ x_3\\ 
\end{smallmatrix*}
\right) 
\rangle=
\langle
\left(
\begin{smallmatrix*}[c]
1\\ 0\\ 0\\ 
\end{smallmatrix*}
\right) 
\rangle $$

Получили, что геометрическая кратность (равна 1) меньше алгебраической кратности (равна 3). Тогда матрица недиагонализируема (не хватило собственных векторов).    