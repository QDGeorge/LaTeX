\subsection{Свойства матрицы Грама}
Свойства матрицы Грама:
\begin{enumerate}
	\item Симметричность.
	\item Положительная ориентированность.
\end{enumerate}
\[
\begin{vmatrix}
\textbf {(e$_1$, e$_1$)} & \textbf {(e$_1$, e$_2$)} \\
\textbf {(e$_2$, e$_1$)} & \textbf {(e$_2$, e$_2$)}
\end{vmatrix}
> 0 \Leftrightarrow \textbf{(e$_1$, e$_2$)}^{2} < \textbf{e$_1^2$ e$_2^2$}
\]
Для линейно независимых векторов $\textbf{x}$ и $\textbf{y}$ $\longmapsto$ $\textbf{(x, y)}^{2} < \textbf{x}^{2} \textbf{y}^{2}.$ \\
Для линейно зависимых векторов $\textbf{x}$ и $\textbf{y}$ $\longmapsto$ $\textbf{(x, y)}^{2} = \textbf{x}^{2} \textbf{y}^{2}.$

Итак,
$$\boxed {\forall~ \textbf{x}, \textbf{y} \longmapsto \textbf{(x, y)}^{2} \le \textbf{x}^{2} \textbf{y}^{2}}$$ --- неравенство Коши-Буняковского-Шварца.
\\\\
Определим длину вектора как $$|\textbf{x}| = \sqrt {\textbf{(x, x)}},$$
а угол между векторами
$$\cos \varphi = \cfrac{\textbf{(x, y)}}{|\textbf{x}||\textbf{y}|}.$$
