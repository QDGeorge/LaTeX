%\documentclass{article}
%\usepackage[utf8]{inputenc}
%\usepackage[cp1251]{inputenc}  
%\usepackage[T2A]{fontenc}      
%\usepackage[russian]{babel} 



%\usepackage{natbib}
%\usepackage{graphicx}
%\usepackage{amsfonts}
%\usepackage{amsmath}
%\usepackage{mathtools}
%\date{}
%\newenvironment{psm}
%{\left(\begin{smallmatrix}}
%	{\end{smallmatrix}\right)}
%\title{Семинар 4. Замена базиса. Сумма и пересечение подпространств.}
%\begin{document}
	
%	\maketitle
	
	\section{Замена базиса.}
	Рассмотрим базис $\overline{e}(e_1,e_2,\dots,e_n)$.
	
	Координаты $x$ в базисе $\overline{e}$: $ \begin{pmatrix}
	\xi_1\\  
	\vdots\\
	\xi_n\\
	\end{pmatrix} = \overline{\xi}$
	
	\vspace{0.25cm}
	Тогда: $x  = \sum\limits_{i=1}^n \xi_ie_i = \overline{e}\overline{\xi}$.
	\vspace{5mm}
	
	Пусть есть два базиса $\overline{e}, \overline{e'}$. Выразим один базис через другой:
	
	\begin{equation*}
	\left.
	\begin{aligned}
	e_1' &= a_{11}e_1 + \dots+ a_{1n}e_n\\
	e_2' &= \dots\dots\dots\dots\dots\dots\\
	e_n' &= a_{n1}e_1 + \dots  + a_{nn}e_n
	\end{aligned} 
	\right\} \Rightarrow e_i' = \sum\limits_{j=0}^n a_{ij}e_j \end{equation*}
	
	
	\vspace{3mm}
	
	$S$ ~---~ \textsf{матрица перехода} от $\overline{e}$ к $\overline{e'}$ $(\det S\neq 0).$
	\begin{equation*}
	\left.
	\begin{aligned}S = \begin{pmatrix}
	a_{11} & \dots & a_{n1}\\
	\hdotsfor{3}\\
	a_{1n} & \dots & a_{nn}\\
	\end{pmatrix}
	\end{aligned} 
	\right. \Rightarrow \boxed{\overline{e'} = \overline{e}S}
	\end{equation*}
	\vspace{3mm}
	
	Связь координат:
	\begin{equation*}
	\left.
	\begin{aligned}
	x&=\overline{e} \overline{\xi}\\
	x&=\overline{e'}\overline{\xi'}=\overline{e}\underbracket{S\overline{\xi'}}_{\xi}\\
	\end{aligned} 
	\right. \Rightarrow \boxed{\overline{\xi} =S\overline{\xi'}}
	\end{equation*}
	\vspace{3mm}
\begin{prim}	
	Доказать, что $F:\begin{psm}
	4\\
	2\\
	1\\
	\end{psm},\begin{psm}
	5\\
	3\\
	2\\
	\end{psm},\begin{psm}
	3\\
	2\\
	1\\
	\end{psm}$ и $G:\begin{psm}
	-1\\
	4\\
	0\\
	\end{psm},\begin{psm}
	4\\
	3\\
	1\\
	\end{psm},\begin{psm}
	1\\
	2\\
	3\\
	\end{psm}$ ~---~ базис в $\mathbb{R}^3$
	\begin{enumerate}
		\item Найти $S$ от $F$ к $G$.
		\item Зная $\overline{\xi'}$ в $G$, найти $\overline{\xi}$ в $F$.
	\end{enumerate}
\end{prim}
	$$G = \begin{pmatrix*}[r]
	-1 & 4 & 1\\
	4 & 3 & 2\\
	0 & 1 & 3\\
	\end{pmatrix*},\quad \det G =-47 \neq 0 \Rightarrow G \text{ --- базис}$$
	\begin{enumerate}
		\item 
		
		$G=FS \hspace{3mm} F^{-1}|\cdot$
		
		$F^{-1}G = S$\\
		Пусть $F$ ~---~ невырожденная, тогда $\exists T_1,\dots,T_n\ (\text{элементарные преобразования матриц})$:
		
		$T_n\dots T_1F=E \hspace{3mm} |\cdot F^{-1}G$
		
		$T_n\dots T_1G=F^{-1}G=S$
		\vspace{2mm}
		
		Т.е. преобразования, которые переведут $F$ в $E$, переведут $G$ в $S$.
		
		$(F|G) \rightarrow (E|S)$
		
		$\left( \begin{array}{ccc|ccc}
		4 & 5 & 3 & -1 &4 & 1  \\
		2 & 3& 2&4&3&2\\
		1&2&1&0&1&3
		\end{array}\right)\rightarrow \left( \begin{array}{ccc|ccc}
		1  & 0 & 0 & -5 &0 & 4  \\
		0 & 1& 0&-4&1&4\\
		0&0&1&13&3&-1
		\end{array}\right)
		\Rightarrow S=\left( \begin{array}{ccc}
		-5 &0 & 4  \\
		-4&1&4\\
		13&3&-1
		\end{array}\right)
		$
		\item $\overline{\xi}=S\overline{\xi'}$
		$$\begin{pmatrix*}[r]
		\xi_1\\  
		\xi_2\\
		\xi_3\\
		\end{pmatrix*} = \begin{pmatrix*}[r]
		-5 & 0 & 4\\
		-4 & 1 & 4\\
		13 & 3 & -1\\
		\end{pmatrix*}\begin{pmatrix*}[r]
		\xi_1'\\  
		\xi_2'\\
		\xi_3'\\
		\end{pmatrix*} =\begin{pmatrix*}[c]
		-5\xi_1+4\xi_3 & \\  
		-4\xi_1+\xi_2+4\xi_3 & \\
		13\xi_1 +3\xi_2 -\xi_3 & \\
		\end{pmatrix*}
		$$
		
	\end{enumerate}
%---------------------------------------------------------------------------
\begin{prim}(Условие то же, что и в примере 1)\\
$
F\!:~~<\!
\begin{psm}
0&1&-2\\
-1&0&3\\
2&-3&0\\
\end{psm}
,
\begin{psm}
	0&0&-1\\
	0&0&4\\
	1&-4&0\\
\end{psm}
,
\begin{psm}
0&-1&2\\
1&0&-2\\
-2&2&0\\
\end{psm}
\!>
$
\quad
$
G\!:~<\!
\begin{psm}
0&1&1\\
-1&0&-1\\
-1&1&0\\
\end{psm}
,
\begin{psm}
0&3&5\\
-3&0&2\\
-5&-2&0\\
\end{psm}
,
\begin{psm}
0&1&0\\
-1&0&3\\
0&-3&0\\
\end{psm}
\!>
$
\end{prim}\\
Заметим, что перед нами кососимметричные матрицы.
\begin{definition}
\textsf{Кососимметричная (кососимметрическая) матрица} --- квадратная матрица $A$, удовлетворяющая соотношению:
$$
A^{\text{T}}=-A \Leftrightarrow a_{ij}=-a_{ji}\ \forall i,j = \overline{1, n}.
$$
\end{definition}
Отсюда следует, что $\dim L'=3$. Базис: $\left\{
\begin{pmatrix*}[r]
0&1&0\\
-1&0&0\\
0&0&0\\
\end{pmatrix*}
,
\begin{pmatrixr}
0&0&1\\
0&0&0\\
-1&0&0\\
\end{pmatrixr}
,
\begin{pmatrixr}
0&0&0\\
0&0&1\\
0&-1&0\\
\end{pmatrixr}
\right\}$\\
Тогда координаты наших векторов $F$ и $G$ в этом базисе:
$$
F
\begin{psm}
1\\-2\\3
\end{psm}
,
\begin{psm}
0\\-1\\4
\end{psm}
,
\begin{psm}
-1\\2\\-2
\end{psm}
\quad
G
\begin{psm}
1\\1\\-1\\
\end{psm}
,
\begin{psm}
3\\5\\2
\end{psm}
,
\begin{psm}
1\\0\\3
\end{psm}
$$
Выполним действия аналогично примеру 1 и получим:
$$
S=
\begin{pmatrixr}
9&40&9\\
-3&-11&-2\\
8&37&8\\
\end{pmatrixr}
$$
\section{Сумма и пересечение подпространств}
Рассмотрим линейные подпространства $L_1$ и $L_2$.
\begin{definition}
	\textsf{Пересечением} $L_1$ и $L_2$ называется множество векторов принадлежащих и $L_1$, и $L_2$.
\end{definition}
\noindent$L_1\cap L_2$ --- линейное подпространство.
\begin{definition}
	\textsf{Суммой} $L_1$ и $L_2$ называется линейная оболочка их объединения.
\end{definition}
\begin{definition}
	Если $L_1\cap L_2 = \{0\}$, то пишут так:	
$$
L_1+L_2=L_1 \oplus L_2,
$$
а сумму называют \textsf{прямой суммой}.
\end{definition}
Если $L=L_1\oplus L_2$, то говорят, что $L_1$ и $L_2$ --- \textsf{прямые дополнения друг друга}.
	
	
	
	
	
	