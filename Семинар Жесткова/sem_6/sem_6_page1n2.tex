Для начала решим небольшой пример по прошлому семинару.
\begin{prim}
$\phi: M_{2\times 2}\rightarrow M_{2\times 1}, \phi(\textbf{x})=\textbf{x}\begin{psm}
	1\\4\\
	\end{psm}$. Найти матрицу линейного преобразования $A$.
\end{prim}\\
Запишем базисы:\\
$M_{2\times 2}: \textbf{e}
\left\{
\begin{psm}
1&0\\
0&0\\
\end{psm}
,
\begin{psm}
0&1\\
0&0\\
\end{psm}
,
\begin{psm}
0&0\\
1&0\\
\end{psm}
,
\begin{psm}
0&0\\
0&1\\
\end{psm}
\right\}\\
M_{2\times1}: \textbf{f}
\left\{
\begin{psm}
1\\0
\end{psm}
,
\begin{psm}
0\\1
\end{psm}
\right\}.
$\\
Для удобства в общем виде найдём, что значит наше преобразование:\\
$$
\phi(\textbf{x})=\begin{pmatrixr}
a&b\\
c&d\\
\end{pmatrixr}
\begin{pmatrixr}
1\\4
\end{pmatrixr}
=
\begin{pmatrixr}
a+4b\\c+4d
\end{pmatrixr}.
$$
Далее <<прогоним>> через преобразование базис \textbf{e}:\\
$
\phi(\textbf{e$_1$})=
\begin{psm}
1&0\\
0&0\\
\end{psm}
\begin{psm}
1\\4
\end{psm}
=
\begin{psm}
1\\0
\end{psm}, \quad
\phi(\textbf{e$_2$})=\begin{psm}
4\\0
\end{psm}, \quad
\phi(\textbf{e$_3$})=\begin{psm}
0\\1
\end{psm}, \quad
\phi(\textbf{e$_4$})=\begin{psm}
0\\4
\end{psm}.
$\\
Отсюда, получаем ответ:
$$
A=\begin{pmatrixr}
1&4&0&0\\
0&0&1&4\\
\end{pmatrixr}.
$$
\section{Рассмотрение ядра и образа}
Рассмотрим $\phi: \underset{\dim L=n}{L}\rightarrow \underset{\dim \overline{L}=m}{\overline{L}}$.\\
Ядро: $\ker\phi: \{\textbf{x} \in L: A\textbf{\textbf{x}}=\textbf{o}\}$\\
Очевидно, что ЛНЗ решения такого уравнения формируют ФСР, а ФСР задаёт линейное подпространство. Вспоминая количество столбцов в ФСР, легко получить формулу:
\begin{equation}
\boxed{\dim\ker\phi=n-\Rg A}.
\label{ker}
\end{equation}
Образ $\im\phi: \{\textbf{y} \in \overline{L}:\exists \textbf{x}\in L: A\textbf{x}=\textbf{y}\}$.\\
Аналогично $\im\phi \in \overline{L}$ формирует линейное подпространство т.к.
\begin{center}
$
A\textbf{x$_1$}+A\textbf{x$_2$}=A(\textbf{x$_1$}+\textbf{x$_2$})$

$A\alpha \textbf{x}=\alpha A\textbf{x}$.
\end{center}
Выберем в $L$ базис $\textbf{e}:\{\textbf{e$_1$}, \textbf{e$_2$}, \dots, \textbf{e$_n$}\}$.\\
$\forall \textbf{x} \in L: \textbf{x}=\alpha_1\textbf{e$_1$}+\dots+\alpha_n\textbf{e$_n$} \leftarrow \phi$ (это обозначение значит <<подействуем преобразованием $\phi$>>)\\
$\phi(\textbf{x})=\alpha_1 \underset{=\textbf{a$_1$}}{\phi(\textbf{e$_1$})}+\dots+\alpha_n \underset{=\textbf{a$_2$}}{\phi(\textbf{e$_n$})}=\langle \textbf{a$_1$},\dots, \textbf{a$_n$}\rangle$.\\
Заметим, что $\textbf{a$_1$},\dots, \textbf{a$_n$}$ --- столбцы матрицы $A$. Отсюда следует формула:
\begin{equation}
\boxed{\dim\im\phi=\Rg A=r}.
\label{im}
\end{equation}
Сложим формулы \eqref{ker} и \eqref{im} и получим:
\begin{equation}
\boxed{\dim\ker\phi+\dim\im\phi=n}.
\end{equation}
\begin{center}
В примерах 2--5: $L=\mathbb{R}^4, \overline{L}=\mathbb{R}^3, A=\begin{psm}
0&0&2&-2\\
2&-4&1&1\\
-1&2&1&-2\\
\end{psm}$.
\end{center}
\begin{prim}
Найти образ $\textbf{x}=\begin{psm}
	1\\1\\1\\1\\
\end{psm}$.
\end{prim}\\
$\phi: A\textbf{x}=\textbf{y}$, т.е. нужно перемножить матрицу $A$ и вектор $\textbf{x}$.\\
$
\begin{psm}
0&0&2&-2\\
2&-4&1&1\\
-1&2&1&-2\\
\end{psm}
\begin{psm}
1\\1\\1\\1
\end{psm}
=
\begin{psm}
0\\0\\0
\end{psm}
=\textbf{o}
\Rightarrow$ ядро не пусто.
\begin{prim}
Найти прообраз $\textbf{y}=\begin{psm}
	4\\0\\3
\end{psm}.$
\end{prim}\\
Итак $\phi: \underline{A}\textbf{x}=\underline{\textbf{y}}$. Мы знаем то, что подчёркнуто. Очевидно, что мы получили СЛУ относительно \textbf{x}. Решим ее.