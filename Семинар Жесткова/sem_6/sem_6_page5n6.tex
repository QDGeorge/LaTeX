Биекция $=$ сюръекция $+$ инъекция\\
$\Rg A=n=m$\\
Т.о. \textbf{биекция задаётся невырожденной матрицей}. В таком случае
$$\dim L = \dim \overline L$$
\textsf{Изоморфизм}~--- биективное линейное отображение.\\
Если существует изоморфизм $L \to \overline L$, то говорят, что $L$ и $\overline L$ \textsf{изоморфны}.

\begin{theorem}
$L$ и $\overline L$ изоморфны$\iff \dim L =
\dim \overline L$.
\end{theorem}

Для изоморфизма $\exists$ $\varphi^{-1}$ --- \textsf{обратное отображение}, его матрица $A^{-1}$.

\begin{prim}
Доказать, что отображение $\varphi(f(x)) = 2f(x) + f'(x)$~--- изоморфизм
в пространстве $P_2$~---  многочленов степени не выше~$2$. Найти $\varphi^{-1}$.
\end{prim}\\

Стандартный базис $L\colon\{ 1, x, x^2 \}$, где $\textbf{e$_1$}=1$, $\textbf{e$_2$}=x$, $\textbf{e$_3$}=x^2$.\\
Общий вид $f(x) = ax^2 + bx + c$,
$\dim L = 3$.
\begin{gather*}
	\varphi(f(x)) = 2ax^2 + 2bx + 2c + 2ax + b =
	2ax^2 + (2a + 2b)x + (b + 2c),
\quad
	\overline L\colon\{ 1, x, x^2 \}
\end{gather*}
$$
\left.
\begin{aligned}
\varphi (e_1)=2
\left(
\begin{smallmatrix}
2\\ 0\\ 0\\
\end{smallmatrix}
\right) \\
\varphi (e_2)=2x+1
\left(
\begin{smallmatrix}
1\\ 2\\ 0\\
\end{smallmatrix}
\right) \\
\varphi (e_3)=2x^2+2x
\left(
\begin{smallmatrix}
0\\ 2\\ 2\\
\end{smallmatrix}
\right) 
\end{aligned}
\right\}
\Rightarrow
A=
\begin{pmatrix}
2 & 1 & 0\\
0 & 2 & 2\\
0 & 0 & 2\\
\end{pmatrix}
$$
$\Rg=3 \Rightarrow$ изоморфизм. Найдем $A^{-1}$:
\begin{gather*}
	A^{-1} = \frac{1}{4} \begin{pmatrix} 2 & -1 & 1 \\ 0 & 2 & -2 \\
	0 & 0 & 2 \end{pmatrix}.
\end{gather*}
Ответ: $\varphi^{-1} \colon A^{-1} = \cfrac{1}{4} \begin{pmatrix} 2 & -1 & 1 \\ 0 & 2 & -2 \\
0 & 0 & 2 \end{pmatrix}$.

 \section{Матрица отображения в новых базисах.}
Пусть в $L$ и $\overline L$ выбраны базисы $\textbf{e}$ и $\textbf{f}$, задано отображение $\varphi\colon L \to \overline L: A$.
Поменяем базисы: $ \textbf{e}' = \textbf{e} S$, $ \textbf{f}' = \textbf{f} P$.
Найдём $A'$:\\
\begin{gather*}
\textbf{x} \in L,~ \textbf{x}=\left(\begin{smallmatrix}
\xi_1 \\ \vdots\\ \xi_n\\
\end{smallmatrix} \right)=\bm{\xi};~~~~~~
\textbf{y} \in L,~ \textbf{y}=\left(\begin{smallmatrix}
\eta_1 \\ \vdots\\ \eta_n\\
\end{smallmatrix} \right)=\bm{\eta}
\end{gather*}
Из теории отображений: $\bm{\eta} = A \bm{\xi};~~~\bm{\eta}' = A' \bm{\xi}';$\\
Из замены базиса: $	\bm{\eta} = P \bm{\eta}';~~~	\bm{\xi} = S \bm\xi';$

\begin{gather*}
	P \bm \eta' = A \bm \xi = AS \bm\xi'
\Rightarrow
	\bm\eta' = P^{-1} A S \bm\xi'=A' \bm \xi'
\Rightarrow
	\boxed{A' = P^{-1} A S.}
\end{gather*}
Если $\varphi$~--- преобразование, то $P=S$ и $A'=S^{-1}AS$\\

\begin{prim} 
	Дано преобразование $\varphi\colon A = \begin{pmatrix} 0 & 2 \\ -1 & -3 \end{pmatrix}$ в базисе $\bf e$. Смена базиса: $\bf e' = \bf e \begin{pmatrix} 2 & 1 \\ -1 & -1 \end{pmatrix}$.
Найти $A'$.\\

Воспользуемся $A'=S^{-1}AS$. Посчитаем $S^{-1}$:\\
$$S^{-1} =
\begin{pmatrix} 1 & 1 \\ -1 & 2 \end{pmatrix}$$.
\end{prim}